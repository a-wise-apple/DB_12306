\documentclass{beamer}
\usepackage[utf8]{inputenc}
\usepackage{ctex} % Support for Chinese characters
\usepackage{listings}
\usepackage{xcolor}
\usepackage{hyperref} % Ensure hyperref is loaded

% Setup hyperref for Chinese bookmarks
\hypersetup{
    pdfencoding=auto,
    unicode=true,
    colorlinks=true,
    linkcolor=black,
    urlcolor=blue
}

% Theme selection
\usetheme{Madrid}
\usecolortheme{default}

% Code listing settings
\definecolor{codegreen}{rgb}{0,0.6,0}
\definecolor{codegray}{rgb}{0.5,0.5,0.5}
\definecolor{codepurple}{rgb}{0.58,0,0.82}
\definecolor{backcolour}{rgb}{0.95,0.95,0.92}

\lstdefinestyle{mystyle}{
    backgroundcolor=\color{backcolour},   
    commentstyle=\color{codegreen},
    keywordstyle=\color{magenta},
    numberstyle=\tiny\color{codegray},
    stringstyle=\color{codepurple},
    basicstyle=\ttfamily\footnotesize,
    breakatwhitespace=false,         
    breaklines=true,                 
    captionpos=b,                    
    keepspaces=true,                 
    numbers=left,                    
    numbersep=5pt,                  
    showspaces=false,                
    showstringspaces=false,
    showtabs=false,                  
    tabsize=2
}

\lstset{style=mystyle}

% Title Page Information
\title{铁路售票系统项目介绍}
\author{陈治杰 曾灵杰 刘子鸣}
\date{\today}

\begin{document}

\frame{\titlepage}

\section{项目需求 (Project Requirements)}

\begin{frame}{项目需求}
    本项目旨在构建一个功能完善的铁路售票系统(类似 12306),支持用户购票、退票、改签以及管理员对车次、车站、时刻表的管理。

    \begin{block}{核心功能模块}
        \begin{itemize}
            \item \textbf{用户模块}: 用户注册、登录、个人信息管理。
            \item \textbf{票务模块}: 车票查询、预订、支付、退票、改签。
            \item \textbf{车次管理}: 列车信息管理、车厢管理、座位管理。
            \item \textbf{时刻表管理}: 列车时刻表制定、经停站管理。
            \item \textbf{运营管理}: 车站管理、员工管理、检票(进站/出站)。
            \item \textbf{交易平台}: 二手票交易或票务转让功能 (\texttt{ticket\_listing})。
        \end{itemize}
    \end{block}
\end{frame}

\begin{frame}{技术栈}
    \begin{itemize}
        \item \textbf{后端}: Java 21, Spring Boot 3.3.5, Spring Data JPA, Spring Security, MySQL.
        \item \textbf{前端}: Vue.js, TypeScript, Vite.
        \item \textbf{数据库}: MySQL 8.0+.
    \end{itemize}
\end{frame}

\section{数据库设计 (Database Design)}

\begin{frame}{数据库设计 - 基础数据表}
    系统采用关系型数据库 MySQL,主要包含以下核心数据表:
    \begin{itemize}
        \item \textbf{station (车站表)}: 存储车站的基本信息(代码、名称、城市)。
        \item \textbf{train (列车表)}: 存储列车的基本信息(车次号、列车类型)。
        \item \textbf{coach (车厢表)}: 定义列车的车厢组成、类型及座位数。
        \item \textbf{seat (座位表)}: 定义车厢内的具体座位信息。
    \end{itemize}
\end{frame}

\begin{frame}{数据库设计 - 调度与时刻表}
    \begin{itemize}
        \item \textbf{train\_schedule (列车时刻表)}: 定义某次列车在具体日期的发车计划及状态。
        \item \textbf{schedule\_stop (经停站表)}: 定义列车时刻表的具体停靠站点、到达及出发时间。
    \end{itemize}
\end{frame}

\begin{frame}{数据库设计 - 订单与交易}
    \begin{itemize}
        \item \textbf{user\_account (用户表)}: 存储用户信息及角色。
        \item \textbf{booking\_order (订单表)}: 记录用户的购票订单及状态(待支付、已支付、已取消等)。
        \item \textbf{seat\_allocation (座位分配表)}: 管理具体时刻表的座位占用情况,防止超卖。
        \item \textbf{ticket (车票表)}: 生成的实际乘车凭证,关联订单、座位及乘客。
        \item \textbf{payment (支付表)}: 记录支付流水。
        \item \textbf{ticket\_listing (票务挂牌表)}: 用于票务转让或二手交易。
    \end{itemize}
\end{frame}

\begin{frame}{数据库设计 - 运营与检票}
    \begin{itemize}
        \item \textbf{employee (员工表)}: 车站工作人员信息。
        \item \textbf{checkin (检票记录表)}: 记录乘客进出站的检票日志。
    \end{itemize}
\end{frame}

\begin{frame}[fragile]{数据库模式 (Database Schema) - 基础信息}
\begin{lstlisting}[language=SQL]
CREATE TABLE station (
  station_id INT PRIMARY KEY AUTO_INCREMENT,
  code VARCHAR(16) NOT NULL UNIQUE,
  name VARCHAR(128) NOT NULL,
  city VARCHAR(128) NOT NULL
);

CREATE TABLE train (
  train_id INT PRIMARY KEY AUTO_INCREMENT,
  train_no VARCHAR(32) NOT NULL UNIQUE,
  train_type VARCHAR(32) NOT NULL
);
\end{lstlisting}
\end{frame}

\begin{frame}[fragile]{数据库模式 (Database Schema) - 时刻表}
\begin{lstlisting}[language=SQL]
CREATE TABLE train_schedule (
  schedule_id INT PRIMARY KEY AUTO_INCREMENT,
  train_id INT NOT NULL,
  depart_date DATE NOT NULL,
  status ENUM('PLANNED', 'OPEN', 'DEPARTED', 'ARRIVED', 'CANCELLED') NOT NULL DEFAULT 'PLANNED',
  CONSTRAINT fk_schedule_train FOREIGN KEY (train_id) REFERENCES train (train_id)
);
\end{lstlisting}
\end{frame}

\section{接口文档 (API Documentation)}

\begin{frame}{接口文档}
    后端提供 RESTful API 供前端调用,主要接口如下:
    \begin{block}{认证模块 (Auth)}
        \begin{itemize}
            \item \texttt{POST /api/auth/login}: 用户登录。
            \item \texttt{POST /api/auth/register}: 用户注册。
        \end{itemize}
    \end{block}
    \begin{block}{预订模块 (Booking)}
        \begin{itemize}
            \item \texttt{GET /api/bookings/user/\{userId\}}: 获取指定用户的订单列表。
            \item \texttt{POST /api/bookings/reserve}: 发起座位预订请求,创建订单。
            \item \texttt{POST /api/bookings/\{id\}/cancel}: 取消指定订单。
        \end{itemize}
    \end{block}
\end{frame}

\begin{frame}{接口文档 (续)}
    \begin{block}{车次与时刻表 (Schedule \& Train)}
        \begin{itemize}
            \item \texttt{GET /api/schedules/search}: 根据出发地、目的地和日期查询车次。
            \item \texttt{GET /api/trains/\{id\}}: 获取列车详情。
        \end{itemize}
    \end{block}
    \begin{block}{车站管理 (Station)}
        \begin{itemize}
            \item \texttt{GET /api/stations}: 获取所有车站列表。
            \item \texttt{POST /api/stations}: 新增车站(管理员)。
        \end{itemize}
    \end{block}
\end{frame}

\section{关键功能代码实现}

\begin{frame}[fragile]{订单服务 (BookingService)}
    \texttt{BookingService} 负责处理订单的核心业务逻辑。
\begin{lstlisting}[language=Java]
@Service
@RequiredArgsConstructor
@Transactional
public class BookingService {
    private final BookingOrderRepository bookingOrderRepository;

    public BookingOrder createOrder(BookingOrder order) {
        order.setStatus(OrderStatus.PENDING);
        return bookingOrderRepository.save(order);
    }
    
    public void cancelOrder(Integer orderId) {
        // ... implementation details
    }
}
\end{lstlisting}
\end{frame}

\begin{frame}[fragile]{订单控制器 (BookingController)}
\begin{lstlisting}[language=Java]
@RestController
@RequestMapping("/api/bookings")
@RequiredArgsConstructor
public class BookingController {
    private final BookingService bookingService;

    @PostMapping("/reserve")
    public ResponseEntity<BookingOrder> reserveSeats(@Valid @RequestBody ReserveSeatsRequest request) {
        BookingOrder order = new BookingOrder();
        return ResponseEntity.ok(bookingService.createOrder(order));
    }
}
\end{lstlisting}
\end{frame}

\begin{frame}[fragile]{时刻表服务 (ScheduleService)}
\begin{lstlisting}[language=Java]
@Service
public class ScheduleService {
    public void createSchedule(CreateScheduleRequest request) {
        // Create Schedule
        TrainSchedule schedule = new TrainSchedule();
        // ...
        trainScheduleRepository.save(schedule);

        // Create Stops, Coaches, Seats
        // ...
    }
}
\end{lstlisting}
\end{frame}

\begin{frame}[fragile]{票务交易服务 (TradingService)}
\begin{lstlisting}[language=Java]
@Service
public class TradingService {
    @Transactional
    public void createListing(Integer userId, Integer orderId, BigDecimal price) {
        // Check ownership and status
        // Create listing
    }

    @Transactional
    public void buyListing(Integer buyerId, Integer listingId) {
        // Transfer ownership
        // Update ticket passenger name
    }
}
\end{lstlisting}
\end{frame}

\section{前端实现 (Frontend Implementation)}

\begin{frame}[fragile]{选座与预订页面 (Booking.vue)}
\begin{lstlisting}[language=JavaScript]
// Load seats and group by coach
const seatsByCoach = computed(() => {
  const groups: Record<string, SeatAllocation[]> = {}
  allAllocations.value.forEach(a => {
    const coachNo = a.seat.coach.coachNo
    if (!groups[coachNo]) {
      groups[coachNo] = []
    }
    groups[coachNo].push(a)
  })
  return groups
})
\end{lstlisting}
\end{frame}

\begin{frame}[fragile]{交易平台页面 (TradingPlatform.vue)}
\begin{lstlisting}[language=JavaScript]
const handleBuy = async (listing: TicketListing) => {
  if (!userStore.user) {
    ElMessage.warning('Please login first');
    return;
  }
  // Confirm and buy logic
  await buyListing(listing.listingId, userStore.user.id);
};
\end{lstlisting}
\end{frame}

\begin{frame}
    \centering
    \Huge 谢谢观看!\\
    \large Q \& A
\end{frame}

\end{document}
